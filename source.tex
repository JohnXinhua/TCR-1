\documentclass[a4paper]{article}

\usepackage[english]{babel}
\usepackage[utf8]{inputenc}
\usepackage{textcomp}
\usepackage{amsmath}
\usepackage{graphicx}
\usepackage{float}
\usepackage{listings}
\usepackage{color}
\usepackage[margin=1in]{geometry}
\usepackage[colorinlistoftodos]{todonotes}

\lstset{
	backgroundcolor=\color{white},
	tabsize=3,
	language=C++,
	basicstyle=\footnotesize\ttfamily,
	frame=lines,
	numbers=left,
	numberstyle=\tiny,
	numbersep=5pt,
	breaklines=true,
	showstringspaces=false,
	keywordstyle=\color[rgb]{0, 0, 1},
	commentstyle=\color[rgb]{0, 0.5, 0},
	stringstyle=\color{red}	
}

\title{Team Code Reference}
\author{
	Timon Knigge
}

\begin{document}
	\maketitle
	\tableofcontents
	\clearpage
	
	\section{Template}
	
	\begin{lstlisting}[language=C++]
#include <iostream>
#include <vector>
#include <stack>
#include <queue>
#include <set>
#include <map>
#include <bitset>
#include <algorithm>
#include <functional>
#include <string>
#include <math.h>

using namespace std;

typedef pair<int, int> ii;
typedef vector<int> vi;
typedef vector<vi> vvi;
typedef vector<ii> vii;
typedef vector<vii> vvii;
typedef long long ll;

int main(){
	int T;
	scanf("%d", &T);
	
	for(int t = 1; t <= T; ++t){
		// solve
	}
	
	return 0;
}
	\end{lstlisting}
	
	\section{Data Structures}
	
	\subsection{Union Find}
	
	\begin{lstlisting}
vii pset;
int psets;

class UnionFind {
private:
	vi parent, rank, setSize;
	int setCount;
public:
	UnionFind(int N) {
		setSize.assign(N, 1);
		setCount = N;
		rank.assign(N, 0);
		parent.assign(N, 0);
		
		for (int i = 0; i < N; ++i) parent[i] = i;
	}
	
	int findSet(int i) {
		return (parent[i] == i) ? i : (parent[i] = findSet(parent[i]));
	}
	
	bool areSameSet(int i, int j) {
		return (findSet(i) == findSet(j));
	}
	
	void unionSet(int i, int j) {
		if (areSameSet(i, j)) return;
		setCount--;
		int pi = findSet(i), pj = findSet(j);
		if (rank[pi] > rank[pj]) {
			parent[pj] = pi;
			setSize[pi] += setSize[pj];
		} else {
			parent[pi] = pj;
			setSize[pj] += setSize[pi];
			if (rank[pi] == rank[pj])) rank[pj]++;
		}
	}
}
	\end{lstlisting}
	\section{Graph Algorithms}
	\section{Computational Geometry}
	\section{Mathematics}
	\section{Helpers}
\end{document}